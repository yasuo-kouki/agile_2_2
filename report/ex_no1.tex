\section*{はじめに}
本研究では,5種類の抵抗に対して画像認識技術を用いた分類を行う.
使用する画像は,各抵抗に印字されたカラーコードが明確に写っているものであり,
これを入力として抵抗の種類を判別することを目的とする.

画像の識別手法としては,
ニューラルネットワークとk-近傍法の2つを用いる.
それぞれの手法について,以下に原理を説明する.

\subsection*{ニューラルネットワークによる識別}
ニューラルネットワークは,人間の脳の神経回路を模倣したモデルであり,
複数の層(入力層,中間層,出力層)から構成される.各層のノード(ニューロン)は,
前の層からの入力に対して重み付けと活性化関数による変換を行い,次の層に信号を伝える.
学習の過程では,入力画像と正解ラベルを用いて誤差を計算し,
その誤差を逆伝播法により各重みに反映させていくことで,分類精度を高めていく.
特に画像認識では,畳み込みニューラルネットワークと呼ばれる構造が広く用いられ,
画像の局所的な特徴(エッジ,色の分布,形状など)を効率的に抽出できるという利点がある.

\subsection*{k-近傍法による識別}
k-近傍法は,教師あり学習におけるシンプルな分類アルゴリズムの一つであり,
新しいデータに対して,既知のデータの中で「距離が近い」k個のデータを参照し,多数決により分類を行う.
ここでの「距離」は通常,ユークリッド距離で定義され,特徴量空間上での近さを表す.
本研究では,画像から抽出された特徴ベクトルを用いて,訓練データとの距離を計算し,
もっとも近いk個のラベルに基づいて,分類先を決定する.
k-近傍法はモデルの学習が不要であり,シンプルである一方,
分類対象の数が多くなると計算量が増加するという特性を持つ.






\section*{はじめに}
本研究は,カラーコードが明確に写った抵抗器の画像を入力として,画像認識技術により抵抗の種類を分類することを目的とする.対象は5種類の抵抗であり,各抵抗に印字された色帯(カラーコード)に基づいて識別を行う.画像から得られる色やパターンといった視覚的特徴を利用することで,人手での測定や読み取りを補助あるいは置換することを目指す.本研究では,識別手法としてニューラルネットワークとk-近傍法を採用し,両者の原理と特徴を明確に示したうえで性能比較を行う.

\subsection*{ニューラルネットワークによる識別}
ニューラルネットワークは,人間の脳の神経回路を模倣した多層の関数近似モデルである.入力層,中間層(隠れ層),出力層から構成され,各ノード(ニューロン)は前層からの信号に重みを乗じ,活性化関数を通して次層へ伝搬することで非線形な変換を行う.学習は,入力画像と対応する正解ラベルを用いて損失関数を定義し,誤差逆伝播法により重みを逐次更新することで進められる.特に画像認識の分野では,畳み込みニューラルネットワーク(Convolutional Neural Network, CNN)が広く用いられ,局所的な特徴(エッジ,色分布,テクスチャなど)を効率的に抽出できるという利点を持つ.本研究では,入力画像に対して前処理を行い,必要に応じてデータ拡張や正規化を施した後,CNNを用いて特徴抽出と分類を同時に学習させる.また,過学習対策としてドロップアウトや正則化,適切な検証データによるハイパーパラメータ調整を行い,汎化性能の向上を図る.

\subsection*{k-近傍法による識別}
k-近傍法(k-Nearest Neighbors, k-NN)は,教師あり学習における非パラメトリックな分類手法であり,新しいサンプルを既知の訓練データとの距離に基づいて分類する.本手法では,あらかじめ画像から特徴ベクトルを抽出しておき,その特徴空間上でユークリッド距離などの距離尺度により訓練データとの近接性を評価する.新規サンプルに対してもっとも近いk個の近傍ラベルの多数決によりクラスを決定するため,学習段階でのパラメータ最適化は不要である.一方で,訓練データの数が増加すると推論時の計算負荷が直線的に増大する点や,特徴量のスケーリングや次元選択が分類性能に大きく影響する点には留意が必要である.本研究では,kの選択や距離尺度の検討,および特徴抽出手法(色ヒストグラムや局所特徴量など)の比較を通じて,k-NNの実用性を評価する.

\subsection*{手法の比較と評価方針}
ニューラルネットワークは自動的に高次元特徴を学習できる一方で,学習に時間と大規模なデータを要する可能性がある.対照的にk-近傍法は実装が容易であり少量データでも有効であるが,特徴設計と計算効率が鍵となる.本研究では,両手法について同一データセットを用いた定量的評価を行い,分類精度,推論速度,計算資源の観点から比較を行う.評価指標としては正解率(accuracy),混同行列による誤識別傾向の解析,必要に応じてF値やクラスごとの適合率・再現率などを用いる.また,誤分類例を分析することで,カラーコードの見え方に起因する誤差要因(照明変化,角度,写り込みなど)を特定し,前処理やモデル改良の方向性を示す.

以上の方針に基づき,本研究は画像認識技術を活用した抵抗識別の有効性を検証し,実運用に向けた実装上の示唆を得ることを目標とする.





